\documentclass{seal_article}

\usepackage{hyperref}
\usepackage{booktabs}

\title{S.E.A.L. Article}
\subtitle{Example}

\begin{document}
\maketitle

\section{Methodology}

\subsection{Bug prediction}
\subsubsection{Project Sample}
Since we used the tool \textsc{ck} \cite{ck} to compute some of our metrics which specifically targets \textsc{Java} projects, we only included Java cloud projects in our sample. Furthermore, we selected projects with at least 100 releases.

\subsection{ASAT adoption in Cloud System}

\subsubsection{Project sample}
We created a sample of 10 cloud projects written in \textsc{Go} that use ASAT and proceeded with the collection as follows: We used \textsc{GitHub}'s search functionality\footnote{https://github.com/search} using query words such as \textit{cloud} to find suitable projects. We filtered the projects by the programming language (\textsc{Go}) and the number of stars (>500). We then manually checked whether a given project really represented a cloud system by reading the repository's description. Finally, we removed projects that do not use ASAT using a script that scans all projects files for ASAT command usage (more details in the next section).



%\begin{lstlisting}[caption=An example code snippet]
%/**
% * Javadoc comment
% */
%public class Foo {
%	// line comment
%	public void bar(int number) {
%		if (number < 0) {
%			return; /* block comment */
%		}
%	}
%}
%\end{lstlisting}

\bibliographystyle{abbrv}
\bibliography{references}

\end{document}
